\documentclass[UTF8]{ctexart}
\usepackage[utf8]{inputenc}
\usepackage[T1]{fontenc}
\usepackage{amsmath}
\usepackage{amsfonts}
\usepackage{amssymb}
\usepackage{mhchem}
\usepackage{stmaryrd}


\title{Nonequilibrium Processes in Physics, Chemistry and Biology }


\author{Sample Questions Part I}
\date{}


\begin{document}
\maketitle


\section{i) Fokker-Plank equation. }

 a) What is the Fokker-Planck equation? 
 
book p39. 
F: conditional prob density.
W:transition rate.
$F=W\Delta t$
 
 $$
\frac{\partial f(p, t)}{\partial t}=\int d k\left[\frac{\partial}{\partial p_{\alpha}}[F(p, k) f(p, t)] k_{\alpha}+\frac{1}{2} \frac{\partial^{2}}{\partial p_{\alpha} \partial p_{\beta}}[F(p, k) f(p, t)] k_{\alpha} k_{\beta}\right]
$$
Introducing notations,
$$
\bar{A}_{\alpha}(p)=\int d k F(p, k) k_{\alpha}, \quad B_{\alpha \beta}(p)=\int d k F(p, k) k_{\alpha} k_{\beta},
$$
we write this equation on the form,
$$
\frac{\partial f(p, t)}{\partial t}=\frac{\partial}{\partial p_{\alpha}}\left[\bar{A}_{\alpha}(p) f(p, t)\right]+\frac{1}{2} \frac{\partial^{2}}{\partial p_{\alpha} \partial p_{\beta}}\left[B_{\alpha \beta}(p) f(p, t)\right]
$$
 
 Another way to express the FP:
 
 $$\frac{\partial f(x, t)}{\partial t} =  \frac{d}{dx}[-a_{1}(x) +\frac{1}{2}\frac{\partial}{\partial x }a_{2}]f(x,t)$$
 
 
 b) Explain the relation between the FPE and the master equation; derive the FPE from the master equation. 
 
  book p38. eq
 
 c) Explain the relation between the FPE and Brownian motion. 
 
 book p38. eq8.12-
 
 
 
 d) What constraint on the FPE coefficients is imposed by an equilibrium environment?
 
 eq8.24


\section{ii) Ch4 Markov process and master equation.}

a) Give the definition and discuss the properties of the Markov stochastic process.

book p16 eq4.8. 
In physics we often deal with some simpler stochastic processes that are called Markov processes. Markov process is defined by relation,
$$
F\left(x_{1}, t_{1} ; \ldots x_{n-1}, t_{n-1} \mid x_{n}, t_{n}\right)=F\left(x_{n-1}, t_{n-1} \mid x_{n}, t_{n}\right),
$$



b) Derive a master equation for Markov processes.

book p17.
eq4.10-eq4.15

c) Apply the master equation to scattering by impurities and derive the collision integral.

p18 eq4.19

d) What is the H-theorem?

Prove the H-theorem for this collision integral. 

eq4.19

Discuss local equilibrium distribution function. 

eq4.26

Explain the relaxation time approximation.

p17

\section{iii) Boltzmann equation.}

a) Give a phenomenological derivation of the Boltzmann equation as a combination of the Liouville equation and the master equation.

p22 eq5.8

b) Derive the particle-particle collision integral from a general master equation.

p24 eq5.11

c) Prove the H-theorem for this collision integral. Derive the equilibrium distribution function. What is the relaxation time approximation?

\section{iv) Boltzmann transport theory.}

a) Explain the principles of Boltzmann transport theory.

b) Calculate the ac conductivity for a gas of charged particles; give physical interpretations of the real and imaginary parts of the conductivity in the frequency domain.
 p.30

\section{v) Langevin equation.}

Consider Brownian motion of a strongly damped particle.

a) Assume a white environmental noise and evaluate the time correlation function $\left\langle v(t) v\left(t^{\prime}\right)\right\rangle$.

b) Assume a colored environmental noise,
$$
\left\langle\xi(t) \xi\left(t^{\prime}\right)\right\rangle=\text { const } \cdot e^{-\left|t-t^{\prime}\right| / \tau},
$$
and evaluate the velocity dispersion $\left\langle v(t)^{2}\right\rangle$. vi) Magneto-resistance.

Calculate magneto-resistance of a dilute gas of charged particles. Use the Boltzmann equation and the relaxation time approximation.

\section{vii) Thermoelectric coefficient.}

Calculate the thermoelectric coefficient of a dilute gas of charged particles.

Use the Boltzmann equation and the relaxation time approximation.

\section{viii) Diffusion.}

Discuss diffusion using various approaches.

a) Derive a diffusion equation using random walk arguments.

b) Derive a diffusion equation from the Boltzmann equation.

c) Discuss the relation between diffusion and Brownian motion.

\section{ix) AC conductivity.}

a) Calculate the ac-conductivity of an ideal gas of charged particles (i.e. linear current response to applied time- dependent electric field) using Boltzmann equation. The relaxation time approximation can be assumed.

b) When is the conductivity purely real? When is it purely imaginary? Give physical interpretations for these cases.

c) Use the derived ac conductivity to evaluate a voltage noise spectral function.

\section{x) Current fluctuations in electrical circuits.}

a) Apply the Langevin approach to calculate the mean square deviation of the current.

b) What is Nyquist-Jonson noise? Calculate the noise spectral density.

xi) Consider a dilute classical gas of ions with charge $q$. Assume that the gas has constant density and vanishing flow velocity. A constant temperature gradient is maintained across the gas. Calculate the electric current density $j$ that appears as a response to the temperature gradient, $j=-\epsilon \nabla T$, assuming the relaxation time approximation.

\section{xii) Classical particle embedded in an environment of linear oscillators.}

a) Explain the physics of Brownian motion, and formulate the model for the environment.

b) Discuss coupling of the particle to the environment.

c) Sketch a derivation of the Langevin equation and discuss connection between damping and fluctuations.

d) Discuss Markovian approximation to the derived Langevin equation.


\end{document}